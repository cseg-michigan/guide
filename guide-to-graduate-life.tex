% Created 2011-08-22 Mon 10:10
\documentclass[11pt]{article}
\usepackage[utf8]{inputenc}
%\usepackage[T1]{fontenc}
%\usepackage{fixltx2e}
\usepackage{graphicx}
\usepackage{longtable}
\usepackage{float}
\usepackage{wrapfig}
%\usepackage{soul}
\tolerance=1000
\usepackage{fullpage}
\RequirePackage{color,graphicx}
\usepackage{hyperref}
\definecolor{linkcolour}{rgb}{0,0.2,0.6}
\hypersetup{colorlinks,breaklinks,urlcolor=linkcolour, linkcolor=linkcolour}
\providecommand{\alert}[1]{\textbf{#1}}

\DeclareUnicodeCharacter{00A0}{~}

\title{CSEG Guide to Graduate Life (2013-2014)}
\author{}
\date{10 June 2013}

\begin{document}

\maketitle

\setcounter{tocdepth}{3}
\tableofcontents
\vspace*{1cm}

\newpage
 \null\vfill
 \noindent
 \copyright 1992-2012, the Regents of the University of Michigan\\
 Produced by the CSE Graduate Student Organization\\
 Original Author: Michael McClennen\\
 Editor of 2009-2010 edition: Benjamin Wester\\
  \\
\\
CSEG\\
2260 Hayward St.\\
Ann Arbor, MI 48109-2122\\
United States of America

\newpage



\section{About this guide}
\label{sec-1}

This guide is the product of the collected experience of the members
of CSEG (Computer Science and Engineering Graduates).  We have
attempted to include as much wisdom from new and old graduate student at
CSE as possible, in this guide. It is not meant to be comprehensive.
Therefore, wherever possible we have tried to point you to official sources of
information.  If you have any questions, just contact any of the CSEG
officers.  Good luck, and welcome to Michigan!

All information in this publication is complete and correct to the
knowledge and understanding of the editor at the time of writing.
Neither the editor nor CSEG nor EECS nor The University of Michigan
nor anyone else assumes responsibility for any anachronism or error
that may be found.  Should the reader find an error, he or she should
contact CSEG for correction in later editions.

This guide can also be found online at
\href{http://cseg.eecs.umich.edu/guide/Guide-2011-12.pdf}{http://cseg.eecs.umich.edu/guide/Guide-2011-12.pdf}.
\subsection{Acknowledgments}
\label{sec-1_1}

The first edition of this Guide was written by Michael McClennen.
The following people made significant contributions to the previous
editions of the Guide:

\begin{itemize}
\item Monica Brockmeyer
\item Andy Poe
\item Ted Tabe
\item Benjamin Wester
\end{itemize}
\newpage
\section{What is CSEG?}
\label{sec-2}

CSEG is the Computer Science and Engineering Graduate student
organization. Every graduate student in the CSE division is
automatically a member. CSEG representatives sit on a number of
departmental committees, acting as liaisons between students and
faculty. We also serve as a focal point for student activities and
interests. Much of our effort goes into organizing social activities
like cookouts, happy hours, and casual sports events, to bring
together students with similar interests and goals. For more
information on CSEG, check out: \href{http://cseg.eecs.umich.edu/}{cseg.eecs.umich.edu}.
\subsection{CSEG Events}
\label{sec-2_1}

Throughout the year, CSEG provides a venue for students to get
together. Frequently, on a Thursday or a Friday, students gather for
happy hours . These happy hours are organized by members of the CSEG community
and are often held in the graduate student lounge on the second floor of the CSE building. At the discretion
of the organizer, they may also be held at an off campus venue. This weekly break provides an opportunity
to meet other students outside of an academic setting, unwind, discuss the events of the week, and enjoy
food and drinks. It also provides a good opportunity for newer and more senior graduate students to
mingle.

CSEG also organizes a department wide Tea Hour every Wednesday afternoon where an assortment of
teas and snacks are served. Unlike the happy hours, these events are attended by students as well as
faculty and provide both groups an opportunity to interact. It also provides a welcome mid-week respite
from work.

CSEG sponsors a variety of recreational sports, including intramural softball, basketball, soccer and
volleyball teams. CSEG owns some sport equipment and encourages graduate students to organize pickup
games of basketball, volleyball, and ultimate Frisbee.

Each semester we have a formal meeting at which our members discuss issues affecting graduate students
and our relationship with the faculty and administration. In past years, we have introduced new faculty
members, talked about departmental financial aid, discussed new course offerings, and talked with members
of the Graduate Employees Organization.

We close out the academic year with another big cookout to celebrate the beginning of summer and the end
of the May qualifying exams. We held last year's cookout outside CSE on north campus, and included food
and games.
\subsection{The 2011-2012 CSEG Board}
\label{sec-2_2}

CSEG is entirely student-run; a new board of officers is elected each year at our winter term meeting. The
board consists of four officers and six representatives to various departmental committees. Professor
Quentin Stout is our faculty advisor. The list of officers and their email addresses is also available on the
CSEG web page.

The entire CSEG board can be reached via email at
\href{mailto:cseg-board@eecs.umich.edu}{cseg-board@eecs.umich.edu}.
President:  Nate Derbinsky
Vice-President:  Shiwali Mohan
Social Chair:  Armin Alaghi
Intramural Chair:  Timur Alperovich
Student/Faculty Tea Chair:  Avishay Livne
Representative to Graduate/Faculty Committee:  Jesh Bratman
Representative to DCO Review Committee:  Mitchell Bloch
Representative to GSAC/UMEC/GSF:  Allie Mazzia


\begin{longtable}{p{6cm}ll}
\caption{CSEG Board Members (2011-2012)} \label{tbl:long}\\
\hline
 \textbf{Name}                                 &  \textbf{Title}            &  \textbf{Email}                   \\
\hline
\endhead
\hline\multicolumn{3}{r}{Continued on next page}\
\endfoot
\endlastfoot
\hline
 Nate Derbinsky                                &  President                 &  \texttt{nlderbin@umich.edu}       \\
 Shiwali Mohan                                 &  Vice-President/Treasurer  &  \texttt{shiwali@umich.edu}        \\
 Rob Cohn                                      &  Secretary/Webmaster       &  \texttt{rwcohn@umich.edu}         \\
 Social Chair                                  &  Armin Alaghi              &  \texttt{alaghi@umich.edu}         \\
 Intramural Chair                              &  Timur Alperovich          &  \texttt{timuralp@umich.edu}       \\
 Student/Faculty Tea Chair                     &  Avishay Livne             &  \texttt{avishay.livne@gmail.com}  \\
 Representative to Graduate/Faculty Committee  &  Jeshua Bratman            &  \texttt{jeshua@umich.edu}         \\
 Representative to DCO Review Committee        &  Mitchell Bloch            &  \texttt{bazald@umich.edu}         \\
 Representative to GSAC/UMEC/GSF               &  Allie Mazzia              &  \texttt{amazzia@umich.edu}        \\
\hline
\end{longtable}
\subsection{Communication}
\label{sec-2_3}

CSEG maintains several e-mail lists.  Incoming students are
automatically put on all e-mail lists.  For others, to subscribe to
any of these mailing lists, send an email to
\texttt{name-request@eecs.umich.edu}, where name is the name of the mailing
list, with \texttt{subscribe} as the subject.  To unsubscribe, send
\texttt{unsubscribe} as the subject.  Should a problem arise, send an email
to cseg-request@eecs.umich.edu.


\begin{longtable}{ll}
\caption{CSEG Mailing Lists} \label{tbl:long}\\
\hline
 \textbf{Email List}  &  \textbf{Purpose}                                    \\
\hline
\endhead
\hline\multicolumn{2}{r}{Continued on next page}\
\endfoot
\endlastfoot
\hline
 cseg                 &  Official CSE department announcements                \\
 cseg-ads             &  Items for sale                                       \\
 cseg-announce        &  General announcements from CSEG                      \\
 cseg-apts            &  Apartments for rent, sublets, looking for roommates  \\
 cseg-disc            &  Unmoderated discussion                               \\
 cseg-seminar         &  Seminar announcements                                \\
 cseg-social          &  Social Events                                        \\
\hline
\end{longtable}


You can also find CSE students chatting over IRC on the \texttt{\#cseg}
channel at \texttt{irc.freenode.org}.
\subsection{Description of CSEG Offices}
\label{sec-2_4}

The president of CSEG is responsible for overseeing major events and
ensuring that the organization is running smoothly.  The president is
also the main contact for students or faculty members who are
interested in CSEG activities.

The vice president is responsible for organizing the fall orientation
for new graduate students and editing this guide.  The vice president
is also the treasurer, overseeing the CSEG budget.

The responsibilities of the secretary include keeping records of
meetings, handling the publicity for special events, and maintaining
our email lists and web page (\href{http://cseg.eecs.umich.edu/}{http://cseg.eecs.umich.edu/}).

The social chairperson is one of our busiest officers.  He organizes
the weekly happy hours and is in charge of our other major social
activities such as the fall and spring cookouts.  Because the job is
an arduous one, the social chair has a committee to assist with happy
hours, cookouts, and pretty much everything involving food.  All
members of CSEG are asked and encouraged to help out the Social
Chairperson by volunteering to host happy hours and cook/setup/clean
at the cookouts.  Having lots of help will ensure that CSEG events
will continue to be successful and fun.

The representative on the CSE faculty committee attends monthly
faculty meetings and provides us with information about the issues
they discuss.

We also have a CSEG representative to the CSE graduate committee and
the executive committee.  They are responsible for broad departmental
policy decisions, making the holder of this position a key liaison
between students and the faculty.

The graduate student forum representative attends meetings with grad
students from other departments to discuss mutual concerns.

The student representative on the Departmental Computing Organization
(DCO) advisory committee is responsible for setting the policies of
DCO.

\newpage
\section{People and Places}
\label{sec-3}

Your activities as a student will carry you to a number of different buildings, and will require you to meet many different people.
\subsection{The Computer Science Building}
\label{sec-3_1}

In January of 2006, the CSE Division (formerly co-located with the ECE Division) moved to its new home on 2260 Hayward St.  The building has two main entrances, one on the north side of the building and one on the south side.  The center of the building is a large open atrium, four stories high, with a glass roof.
\subsubsection{Department Offices}
\label{sec-3_1_1}

This is a list of several people you will contact during the course of
your stay here.

\begin{longtable}{llll}
\caption{Important People at CSE} \label{tbl:long}\\
\hline
 \textbf{Name}                  &  \textbf{Title}                &  \textbf{Phone}  &  \textbf{Email}                 \\
\hline
\endhead
\hline\multicolumn{4}{r}{Continued on next page}\
\endfoot
\endlastfoot
\hline
 Dawn Freysinger                &  Graduate Student Coordinator  &        647-1807  &  \texttt{dawnf@eecs.umich.edu}   \\
                                &  Financial Aid Officer         &                  &                                  \\
 Prof. John P. Hayes            &  Graduate Program Chair        &        763-0386  &  \texttt{jhayes@eecs.umich.edu}  \\
 Prof. Marios C. Papaefthymiou  &  Chair, CSE                    &        764-1260  &  \texttt{marios@umich.edu}       \\
 Prof. Karem Sakallah           &  Associate Chair, CSE          &        764-6894  &  \texttt{karem@umich.edu}        \\
 Karen Liska                    &  Human Resources Coordinator   &        647-4255  &  \texttt{liska@eecs.umich.edu}   \\
\hline
\end{longtable}
\subsubsection{Graduate Student Lounge}
\label{sec-3_1_2}

The Graduate Student lounge is located on the second floor of the CSE building right by the long straight staircase.  Among other things, there is a foosball table and refrigerator here that students can use.  CSEG happy hours are sometimes held here, and students often come here to relax or take a break.
\subsubsection{Entrances}
\label{sec-3_1_3}

All entrances to the CSE building are locked from 7 P.M. until 7 A.M. and the entire weekend.  If you have an office in CSE, you can enter during these hours by swiping your current student ID (Mcard) through card readers posted at the main entrances.  If you do not have an Mcard, or if you have trouble entering the building using your card, visit one of the registrar’s offices for assistance.  Elevator use during these hours also requires you to swipe your Mcard.
\subsubsection{Mail Rooms}
\label{sec-3_1_4}

Mailrooms are located on the second, third, and fourth floors for the CSE building near the spiral staircase.  Every student has a mail folder on the third floor.  Please check this folder periodically.  If you are part of a lab, the lab may also have a mailbox on a different floor, typically the floor where the lab is located.
\subsubsection{Tishman Hall}
\label{sec-3_1_5}

Tishman Hall is the main lobby area of the CSE building, where one can almost surely find a group of students transfixed on the screen displaying their Nintendo Wii game, or studying at a table by the café.
\subsubsection{Additional Rooms of Interest in EECS}
\label{sec-3_1_6}

The EECS building (1301 Beal Avenue) was the old of home of the CSE division and continues to house certain relevant groups, items, and offices of interest.
\subsubsection{ACM Office}
\label{sec-3_1_7}

It is located at 1219 EECS building. You can pick up an ACM membership form here if you are interested in joining ACM.
\subsubsection{IEEE Office}
\label{sec-3_1_8}

It is located at 1236 EECS building. The Institute of Electrical and
Electronics Engineers (IEEE) and the Association for Computing
Machinery (ACM) sell bagels, juice and coffee every morning outside
the IEEE office.  The office is located at the far east end of the
atrium.  IEEE also has a fax service.  They have membership forms if
you are interested in joining IEEE.
\\
\\
\emph{Important note regarding building safety}:\\
Make sure to familiarize yourself with the exit routes in the building.  When an alarm goes off, no matter what time of the day or night, leave the building immediately.  Some of the research labs (particularly the chip fabrication lab in EECS) deal with highly toxic and flammable materials.
\subsection{The Pierpont Commons}
\label{sec-3_2}

Located at the center of North Campus, at 2101 Bonisteel Blvd., the Pierpont Commons is the closest thing we have to a student center.  There are quite a few useful services located here, including:

\begin{itemize}
\item The North Campus Information Center, which features a bulletin board
  and a person on duty during business hours to answer questions about
  campus life.
\item A bookstore, carrying textbooks for most North Campus classes.
\item A registrar’s office, where you can get your student ID card (and also arrange to be on the meal plan if you are living in a dormitory).
\item A branch of the University of Michigan Credit Union.
\item A quiet lounge for studying.
\item A piano lounge to play or listen to music.
\item Automatic Teller Machines.
\item Cafeteria (second floor).
\item Panda Express, Sushi by Panda, Quizno’s Subs, and Beanster’s Coffee
\item U-Go’s convenience store.
\item A pool table and TV lounge.
\item UM Computer Showcase, for discounted computer hardware, software, and accessories.
\end{itemize}

While the scope of activities cannot hope to compare to the Michigan Union on Central Campus, the Commons has in the last year hosted a series of art exhibits, several poster and T-shirt sales, and evening concerts.
\subsection{Buildings Adjacent to CSE}
\label{sec-3_3}

The CSE building and the Commons face each other at opposite ends of
the North Campus Green (also called the “Diag”).  The EECS building
lies at 1301 Beal Ave., to the west of the CSE building.  The EECS
building houses CAEN computer labs and several classrooms that are
used for CS classes and discussions.  The Herbert H. Dow Building,
kitty-corner to EECS at 2300 Hayward St., is the home of chemical
engineering and material sciences.  The Dow building also contains
several large auditoria on the ground floor which are sometimes used
for CSE classes.  The G. G. Brown Laboratory is at 2350 Hayward St.
On the lower floor of G. G. Brown is the Blue Lounge (GGB 1040), which
has several vending machines and is used as a study space.

The Duderstadt Center (formerly called the Media Union, located at
2281 Bonisteel Blvd.) is adjacent to the Chrysler Building and
Pierpont Commons.  It houses the Engineering Library, several
multimedia classrooms, auditoria, instructional labs, and some parts
of the art and music departments, and it is just across the quad from
CSE.

The Lurie Building is located to the south of EECS behind the auto
lab.  It contains offices for the Engineering Deans and conference
rooms.  There is a large reflecting pool in front with benches, which
is a nice place to sit in good weather.
\subsection{Central Campus}
\label{sec-3_4}
\subsubsection{Michigan Union}
\label{sec-3_4_1}

The cultural center is for the students of the University.  This building contains the offices of student organizations, a billiards room, a Ticketmaster outlet, and fast food restaurants in the basement.  There is a Campus Information Center on the first floor. (530 S. State St.)
\subsubsection{Literature, Science and the Arts Building}
\label{sec-3_4_2}

This building is next to the Union, and contains the Registrar's office and the Cashier's office. (500 S. State St.)
\subsubsection{Student Activities Building}
\label{sec-3_4_3}

This is located behind the LS \& A building. The student housing office is on the second floor. (515 E. Jefferson St.)
\subsubsection{Graduate and Undergraduate Libraries}
\label{sec-3_4_4}

The main libraries are located on the south side of the Diag.  There are two:  the Harlan Hatcher Graduate Library (a.k.a. the “Grad library,” 920 S. University Ave.), where the bulk of the University collection is kept, and the Shapiro Undergraduate Library (a.k.a. the “UgLi”).  The Shapiro Library building also houses the Shapiro Science Library.  If you are a Theory student, you will probably be almost as familiar with this library as you are with the Engineering Library.
\subsection{Organizational Structure}
\label{sec-3_5}

The EECS department is composed of two divisions:  Computer Science
and Engineering (CSE), and Electrical and Computer Engineering (ECE).
The EECS department has three distinct Graduate Programs:  Computer
Science and Engineering (CSE), Electrical Engineering (EE), and
Electrical Engineering: Systems. Professor Farnam Jahanian is the CSE
Chair with Professor Mike Wellman as CSE Associate Chair.  The ECE
interim department chair is Professor Brian Gilchrist, and the ECE
Associate Chair is Professor Jeffrey Fessler.  The EECS department is
part of the College of Engineering, which is under the direction of
Dean David C. Munson, Jr.

All graduate degrees at the University of Michigan are granted by a
part of the University called the Horace Rackham School of Graduate
Studies.  This is usually referred to as “Rackham” or “the Graduate
School.”  Rackham sets part of the requirements for graduate degree
programs.  Each department takes these as a minimum and adds to them.
The main offices of the Graduate School are in the Rackham building on
Central Campus.

Like all graduate students at the University, you will register as a
Rackham student.  You even get to elect representatives to the Rackham
Student Government.  As a member of the EECS department, you are also
part of the College of Engineering.

\newpage
\section{Dealing with the University}
\label{sec-4}

The University of Michigan is one of the largest public universities
in the country.  As such, it has a huge weight of bureaucracy that can
sometimes be difficult to deal with.  The EECS department tries to
shield us from this as much as possible but there are still times at
which you will have to interact directly with the University.  Here is
some miscellaneous advice:
\subsection{Mailing Address}
\label{sec-4_1}

As soon as you find a place to live (and every time you move) don't
forget to notify the University about your new address. Do this on the
Wolverine Access website \href{http://wolverineaccess.umich.edu/}{http://wolverineaccess.umich.edu/}.  Note:
don't list a permanent address back home, unless you really plan to go
home every summer.  If you list a permanent address different from
your local address, the University will send its mail to you there
during the summer.

Whenever you change your address, you should also give your new
address to your Graduate Student Coordinator.
\subsection{Getting Money from the University}
\label{sec-4_2}

If you will be getting money from the University for any reason (such
as getting a scholarship, or being a GSI or a GSRA) then you have two
choices as to how to receive the money.  First, you can have the
University deposit payments directly into your bank account.  This is
called direct deposit.  The form is available in Wolverine Access.
Second, you can have the check mailed to your home address.  Likewise,
the form for authorizing a check to be mailed is available in
Wolverine Access.
\subsection{Health Care}
\label{sec-4_3}

If you are going to be a GSI or a GSRA with an employment fraction of
at least 0.25, or if you have a department designated eligible
scholarship or fellowship, you will get health benefits.  The
University pays for these from September to April, and it has been the
department's practice to continue them over the summer for returning
GSRAs and guaranteed financial aid students.  This extra summer
coverage is not mandated, and the department could change its policy.
Make sure that you check with your lab coordinator (if you are a
GSRA), Karen Liska (GSI), or Dawn Freysinger (fellowship recipient) to
ensure that you are covered during the summer before April 1st.

If you are a GSI, you will have the chance to pick from a few plans
that are offered to all University employees.  By default, only you
will be covered under a plan known as GradCare (through the Blue Cross
Network).  This plan is not as complete as others, but it still
provides coverage for hospital care and doctor visits (you pay a fixed
fee “co-pay” each visit, and all other costs are covered by the plan).
It is important that you act without delay to enroll your dependents
or select an alternate plan.  If you miss the 30-day deadline, you
cannot enroll your dependents or change your insurance until a year
from next January!  You will receive an e-mail from the Benefits
Office regarding your enrollment options once your appointment is in
the system.

If you are a GSRA, you do not get to choose your plan:  you will be
covered under GradCare.  By default, this is a single-person plan
covering only you.  It is important that you act without delay to
enroll your dependents.  If you miss the 30-day deadline, you cannot
enroll your dependents until a year from next January!

Even if you are covered on your parents’ insurance, do not decline the
coverage!  It is free to you, and if you are dropped from your parents
insurance, it is very difficult to enroll in a University plan outside
the enrollment period.  You will receive an e-mail from the Benefits
Office regarding your enrollment options once your appointment is in
the system.

If you are on a fellowship, you may or may not be eligible for
GradCare (depending upon which fellowship you hold.)  Check with Dawn
Freysinger to find out.  If you are eligible, you will be enrolled
automatically and will receive a confirming e-mail from the Benefits
Office.  Watch for it and contact Dawn if you don’t receive it.

If you are on a non-eligible fellowship, or if you are paying your own
way, you are responsible for your own health care.  If you are under
25 years old, you may be eligible for coverage under your parents'
plan.  If you are working part-time, you may be able to arrange health
benefits through your job.  As a last resort, the Michigan Student
Assembly has arranged for a health plan that students may purchase.
It does not give very good coverage, but is better than nothing.
\subsubsection{Health Plan Options}
\label{sec-4_3_1}

The Benefits Office contains information on the exact plans available
to graduate students.  The relevant information is located at
\href{http://www.umich.edu/~benefits/jobgroups/grads.htm}{http://www.umich.edu/\~benefits/jobgroups/grads.htm}.

Make sure that you fully understand your health care options.  If you
are a GSI or GSRA, and if the staff in the Benefits Office cannot
answer your questions, ask Karen Liska (3709 CSE).  Verify that your
coverage is active by reviewing your pay stub each month.
\subsubsection{University Health Services}
\label{sec-4_3_2}

If you just have a cold, or something else minor that won't require
extensive treatment, you can go to University Health Services.  This
is free to all students.  However, you may have to wait in line and
you have to take whichever doctor happens to be available.  They also
offer prescription drugs at a discount.

\newpage
\section{Academic and Research Plans}
\label{sec-5}

Perhaps the hardest part of life as a new graduate student is the lack
of organized guidance.  Many students find themselves for the first
time in their lives completely responsible for planning and scheduling
their progress.  Academic requirements, research, and exploration of
your new environment will vie with each other for a limited number of
hours in each day, while long-term planning must take into account
many confusing and conflicting goals.

Many sources of guidance and support are available to you.  However,
such support must be actively sought out.  Everyone assumes that
graduate students are able to take care of themselves and that they
know when to ask for help.  Sometimes no one notices that a student is
in trouble until he or she is hopelessly behind.  This may sound
menacing, but it is also true that the department contains lots of
people who are perfectly willing to give you advice and guidance if
you just ask.
\subsection{What to Expect During Your First Year}
\label{sec-5_1}

Being a graduate student is quite different from being an
undergraduate.  In general, you’ll find that there are many more
demands on your time, and that you will be expected to do a lot more
thinking for yourself and a lot less busy work.  Managing your time
wisely and getting accustomed to this new sort of education will be a
major adjustment.

Most students experience a lot of uncertainty and fear in their first
year.  If you’re coming from a smaller school, the size of the EECS
department can be intimidating.  Also, you’re surrounded by a lot of
other really bright folks.  It’s easy to feel inferior or to think
that you don’t belong.  Don’t worry; everyone has this feeling at some
point.  First of all, always keep in mind that it’s not a competition
between you and other graduate students.  You are all here to do
research and get a Ph.D. or M.S.  If you see other students as
potential allies and resources, rather than competitors, life will be
easier.  Also, don’t assume that every other student has the answers.
Students in EECS classes tend not to ask questions, but instead just
sit and nod, even if they’re not following the lecture.  If something
isn’t clear, it’s probably because it wasn’t explained well.  So,
raise your hand and ask.  Not only will you get a better explanation,
all the other students in class will thank you for asking the question
they were afraid to ask.

You’ll also find that graduate courses (500 and above) tend to be more
thought-oriented than 400-level classes.  Rather than doing lots of
homework, often you’ll read papers and do a final project.  The
emphasis here is not on learning of well-known techniques, but
critical analysis of current research in a particular field.  While
the sheer number of hours spent on 500-level classes is usually less,
the intellectual demands are often new to first-year students.  The
skills learned in these classes (reading and synthesizing papers,
evaluating hypotheses, thinking about unsolved problems) are essential
for doing your own research.

During your first year, you’ll be looking for an advisor and a
research program.  (More on this below.)  Don’t feel that you have to
know exactly what your dissertation will be about.  You have a long
time to adjust and refine this.  Instead, focus on finding an area
that has some questions and issues you find interesting.  Also, when
selecting an advisor, make sure you think about whether this is a
person you can get along with.  You’ll be spending a lot of time with
this person over the next few years, and depending upon them for a lot
of guidance and advice, so you want it to be someone you feel
comfortable talking to and receiving advice from.

Most students do an independent research project during the summer
after their first year. Often, this work is the basis of a prelim
paper.  During your second semester, it’s a good idea to start
thinking about what this project might be and talking with potential
advisors.  If you know the area you want to focus on, it can be
helpful to take a 500-level class in that area and use the class
project as a springboard for your own research.  Keep in mind that a
summer project is probably not going to be very big; you’re not going
to create a computer that passes the Turing test in 2 months, and no
one expects you to.  The purpose of this project is to learn how to:
(1) formulate a hypothesis or a research question; (2) determine a
means for testing this hypothesis or answering this question, and (3)
evaluate and explain your results.  The point is not that the results
are earth-shattering, but that you are able to get through the
process.  Faculty view this as a microcosm of the two or three years
you’ll spend working on your dissertation.

Finally, graduate school can be very stressful.  Your first year will
probably be the worst.  You’ll have lots of uncertainty about your
life, lots of work to do, and quite a few demands on your time.
Managing your schedule and dealing with stress are very important,
both to succeeding in your program and keeping your sanity.  You will
often find yourself spending late nights working to get things
finished, especially if you are a GSI.  Being a GSI is a very
rewarding experience, but it can also place a lot of demands on your
time.  Try not to procrastinate; things will always take more time
than you expect.  Also, it’s very important to not let school
completely consume your life. (It’ll take up a pretty big chunk.)  Try
to do something fun (athletics, music, social activities) every week
to relieve stress.  If necessary, schedule this the same way you would
a class.  It’ll help you keep your focus and maintain perspective.
Also, it can be a means of meeting new people.
\subsection{Asking for Help}
\label{sec-5_2}


\begin{itemize}
\item Your advisor: often the best person to ask about research and coursework.
\item Other faculty members: be prepared to get a different opinion from each professor you ask.
\item The Graduate Student Coordinator:  Dawn for CSE.  Always a good person to ask about schedules and requirements.
\item Your peers:  hang around the department, introduce yourself to
  students and ask them questions.  If you are in CSE, go to the CSEG
  peer counseling session and orientation.  We've been here for a
  while, and most of us are glad to tell you about graduate life from
  the inside track.
\end{itemize}
\subsection{Advisor, Who?}
\label{sec-5_3}

Incoming students are each assigned a professor in their area of study
to be their academic advisor.  Your academic advisor will help you
choose your first semester classes.  Talk with your advisor before you
make your final decision of which courses to register for.  Once you
have decided on a research advisor, they will also likely function as
your academic advisor as well.

Advice from current students: the most important criteria in picking a
research advisor are whether you can get along with them and whether
they respect you and your goals.  Don't worry so much about a
prospective advisor's exact area of research.  Professors are almost
always interested in more than they personally have time to
investigate.  Likewise, don't let money be an overriding concern.  If
you can find a good topic to work on, your advisor will be able to
apply for a grant.

Finally, remember that you do not have to make a long-term commitment
to any faculty member at the start.  If you think someone's work is
interesting, try doing a “directed study” (EECS 599) with them for a
semester.  If things work out, you can keep going.  Most students say
that they either knew who their long-term advisor was going to be
before they started school here, or else they made their decision
sometime between the end of their first year and the end of their
second year.  Several ended up switching to a different advisor after
one or two years.  Most ended up switching research topics at least
once.
\subsection{Planning Your Program}
\label{sec-5_4}

How you schedule your time is a personal decision and a very important
one.  Here are some considerations offered by current students.
Different people have differing opinions about the relative importance
of each of these.

\begin{enumerate}
\item Which program?  Even if you plan to stop with a Master’s degree, it
   is a good idea to keep the doctoral program requirements in mind.
   A significant number of students change their mind and continue on
   for their Ph.D.  The Master’s degree requirements are a subset of
   the requirements for qualification in the doctoral program, but the
   latter are somewhat more stringent.
\item Get requirements out of the way early.  Many people think that this
   is a good idea.
\item Patch holes in your knowledge of computer science.  This often
   requires taking some 400-level classes, aimed at both undergrads
   and beginning grad students.  These classes usually involve a lot
   of work, including programming assignments and problem sets.
\item Move towards research.  The 500-level classes are oriented towards
   in-depth study and preparation for research.  They tend to require
   less “grinding” and more insight than undergraduate courses.  Most
   of these courses are offered only once a year, some once every
   other year.  Most of them have 400-level prerequisites, which you
   can skip if you have already taken the equivalent classes.  Ask
   your peers and your advisor for guidance here.
\item Start research early.  If you can find a topic that interests you
   (not necessarily the one on which you'll do your thesis) and a
   professor you can work with, go for it!  After all, that's what
   you're here for, right?
\item Avoid burnout.  Advice from current students is:
\begin{itemize}
\item Don't take more than one course with a large project during any
     given semester.  Watch out for 427, 470, 472, 482, 483, 487, 627.
\item Don't take more than two serious classes if you're a GSI or GSRA,
     or three in any case.  You may have to bend this rule a little
     bit, but be prepared for a tough semester.
\item If you are a GSI or GSRA with a 50\% appointment, expect to work
     at least 20 hours each week.
\end{itemize}
\end{enumerate}
\subsection{Registering for Classes}
\label{sec-5_5}

In general, you should register early.  Each course has an enrollment
limit, with all subsequent registrants placed on a “wait list.”  If
you are taking only EECS courses, then don't panic:  most professors
in our department have a policy of giving overrides for all graduate
students, enabling them to get into the course.  However, if you are
taking courses from other departments you may have trouble getting in
if you wait too long.

The University will charge you a \$50 penalty fee if you register after
classes have started—so don't wait that long!  If you haven't been
able to see your advisor before classes start, just sign up for one or
two classes you think you might want to take.  You can easily change
your schedule any time before the add/drop deadline.

Register for classes online using Wolverine Access
(\href{http://wolverineaccess.umich.edu/}{http://wolverineaccess.umich.edu/}).  In order to register, you must
have a uniqname and password, which are required to log onto the
computer system.

It is very simple to add and drop courses during the first three weeks
of class.  In fact, it is a good idea to register for more courses
than you plan to take, and drop whichever ones you are less interested
in after the first few class meetings.  If you find out about a good
course after classes have already started, sit in on it for a few
meetings and ask the professor for a copy of the syllabus.  Usually,
it is easy to catch up.  The drop/add deadline is three weeks after
classes start, so you should make all of your course decisions by
then.

Most classes on North Campus begin at 10 minutes after the half-hour
(for example, 8:40 and 9:40).  Most classes on Central Campus begin at
10 minutes after the hour (for example, 8:10 and 9:10).  This is known
as “Michigan Time”.  The class times are staggered this way so that
there time to travel between Central Campus and North Campus between
classes.  In my experience, ten minutes is not sufficient time to make
it from North Campus to Central Campus, so don’t schedule classes too
closely.
\subsection{Online Academic Information}
\label{sec-5_6}

Many parts of the University offer on-line information access to
students. Here is a selection:

\begin{longtable}{lp{10cm}}
\caption{Online Information} \label{tbl:long}\\
\hline
\hline
 U-M               &  \href{http://www.umich.edu/}{http://www.umich.edu/}                                                                                                  \\
                   &  Web page for the University of Michigan.                                                                                                             \\
\hline
 Rackham           &  \href{http://www.rackham.umich.edu/new_students/}{http://www.rackham.umich.edu/new\_students/}                                                       \\
                   &  Information for new graduate students.  Graduate Student Handbook. Academic calendar.                                                                \\
\hline
 EECS              &  \href{http://www.eecs.umich.edu/}{http://www.eecs.umich.edu/}                                                                                        \\
                   &  EECS Department web page. Contains information about the courses, activities, people, offices, and laboratories that make up the EECS Department.    \\
\hline
 CSE               &  \href{http://www.cse.umich.edu/}{http://www.cse.umich.edu/}                                                                                          \\
                   &  Contains information on the undergraduate and graduate CSE curriculum.                                                                               \\
\hline
 CSEG              &  \href{http://cseg.eecs.umich.edu/}{http://cseg.eecs.umich.edu/}                                                                                      \\
                   &  The CSEG web page has pointers to some useful information about how to succeed in graduate school. The on-line version of this guide is also there.  \\
\hline
 ECE               &  \href{http://www.eecs.umich.edu/ece/}{http://www.eecs.umich.edu/ece/}                                                                                \\
                   &  Contains information on the undergraduate and graduate ECE curriculum.                                                                               \\
\hline
 MIRLYN            &  \href{http://mirlyn.lib.umich.edu/}{http://mirlyn.lib.umich.edu/}                                                                                    \\
                   &  The University library catalog.                                                                                                                      \\
\hline
 Wolverine Access  &  \href{http://wolverineaccess.umich.edu/}{http://wolverineaccess.umich.edu/}                                                                          \\
                   &  Access your class schedule, get transcripts, or change your address.                                                                                 \\
\hline
\hline
\end{longtable}
\subsection{How to be a Successful Graduate Student}
\label{sec-5_7}

There are several on-line guides written by graduate students that give advice on how to have a successful graduate school experience.  Here are a few:\\
\emph{How to Succeed in Graduate School}\\
\href{http://www.cs.umbc.edu/~mariedj/papers/advice-summary.html}{http://www.cs.umbc.edu/\~mariedj/papers/advice-summary.html}\\
\emph{Information for graduate students \& those considering graduate study}\\
\href{http://www.cs.umbc.edu/csgradinfo.html}{http://www.cs.umbc.edu/csgradinfo.html}\\
\emph{Graduate Student Survival Guide}\\
\href{http://projects.ischool.washington.edu/wpratt/survive.htm}{http://projects.ischool.washington.edu/wpratt/survive.htm}\\
\emph{Advice on Research and Writing}\\
\href{http://www.cs.cmu.edu/afs/cs.cmu.edu/user/mleone/web/how-to.html}{http://www.cs.cmu.edu/afs/cs.cmu.edu/user/mleone/web/how-to.html}\\
\newpage
\section{Avenues of Communication}
\label{sec-6}

Unlike the undergraduate curriculum, the majority of graduate
education takes place outside of the classroom.  Learning often occurs
through interaction with faculty members, fellow graduate students,
and books.  A good introduction to the opportunities for research and
learning can be had by attending the various seminars, discussion
groups, and visiting lectures that are offered each week.
\subsection{Seminars and Discussion Groups}
\label{sec-6_1}

Seminars tend to be informal lectures, usually ending with a question
and answer session.  Abstracts are usually provided in advance.
Seminars are open to everyone, whether or not they have any background
in the subject area.  Discussion groups tend to be a more private, and
are typically composed of a group of graduate students working on the
same project.  If a particular professor is working on a topic in
which you are interested, they will probably be happy to have you sit
in on one of their “working groups” if you inquire ahead of time.

The department plays host almost every week to one or more visiting
scholars or industry representatives who give presentations about
their research.  Some of these are candidates or potential candidates
for faculty positions.  Their lectures are open to the public, and
provide an opportunity to become acquainted with some of the research
work going on outside the department.

Following is a partial list of regular seminars.  For more
information, check the bulletin boards, located outside of the
Graduate offices on the third floors of the EECS building and the CSE
building, and the CSE web site (\href{http://www.cse.umich.edu/}{http://www.cse.umich.edu/}).  These are
updated every Monday to show all of the scheduled lectures and
seminars for the upcoming week.  Note that all of these events run on
“Michigan time,” just like class meetings.  This means that they
actually start 10 minutes after the hour.

Seminar series marked with an asterisk on the list below may be taken as courses for 1-3 credits.

\begin{itemize}
\item EECS 770 (ACAL seminar)
\item EECS 880 (Software seminar)
\item EECS 874 (Theory)
\item AI seminar -- contact Cindy Watts (cwatts@eecs.umich)
\item ITS (Intelligent Transportation Systems) Speaker series -- contact James Rogers (jlrogers@umich.edu)
\end{itemize}
\subsection{Email Lists}
\label{sec-6_2}

Much of the communication within the department takes place via
electronic mail.  It is possible to find out about many of the
seminars and lectures mentioned above by getting your name placed on
various mailing lists.  Much of the communication within the CSE
division goes through the CSEG mailing lists, detailed 3.  Shown below
are some of other lists of interest as well as the e-mail contact
necessary to get your name on each.  Anyone may send mail to any of
these lists, but please keep it relevant to the intended purpose.


\begin{longtable}{lll}
\caption{Other Important Email Lists} \label{tbl:long}\\
 \textbf{List Name}  &  \textbf{Contact}  &  \textbf{Audience or Purpose} \\
\hline
\endhead
\hline\multicolumn{3}{r}{Continued on next page}\
\endfoot
\endlastfoot
\hline
 eseg                &                    &  Announcements to EE students  \\
\end{longtable}
\section{Computing Facilities}
\label{sec-7}

By some accounts, the University of Michigan has some of the best
computing facilities of any university in the country.  Whether or not
this is true, there are several different computing environments
available to you.
\subsection{Computing Organizations}
\label{sec-7_1}

There are three major computing support organizations with which you
will have contact: CAEN, DCO, and ITCS.
\subsubsection{DCO}
\label{sec-7_1_1}

Most of the machines in the EECS department research labs are
maintained by DCO (Departmental Computing Organization).  DCO is a
part of the EECS department.  This is a much smaller organization than
CAEN, and usually easier to deal with.  They maintain approximately
twenty public Linux servers.  These machines are available for remote
login and they handle e-mail, file, and compute service for the
department.  Information can be found at
\href{http://www.eecs.umich.edu/dco/}{http://www.eecs.umich.edu/dco/}.
\subsubsection{ITCS}
\label{sec-7_1_2}

Computing support for the rest of the university is provided by ITCS
(Information Technology Central Services).  The current ITCS was known
as ITD until recently, and this may lead to some confusion when
talking to people who have been here awhile.  You are eligible for an
account with them (useful if you need more disk space), and they also
run the X.500 directory service for the entire University (see
\href{http://directory.umich.edu/}{http://directory.umich.edu/}).  ITCS’s webpage is
\href{http://www.itcs.umich.edu/}{http://www.itcs.umich.edu/}.
\subsubsection{Uniqnames}
\label{sec-7_1_3}

Before you can get a computer account on any of these systems, you
must first get your uniqname. This name will be unique throughout the
University, and will be your e-mail address and login name on each of
the computing systems you will be using.  Your uniqname will be chosen
by ITCS to be some combination of letters from your first and last
name.  It will be printed on your student ID (Mcard) along with your
full name.
\subsubsection{Computing Environment}
\label{sec-7_1_4}

There are many different computing environments at your disposal. You will probably have reason to access all of these during your stay here.
\begin{itemize}

\item DCO workstations\\
\label{sec-7_1_4_1}%
If you join one of the EECS department's research laboratories, the
workstations in the lab will likely be maintained by DCO.  Linux
(Fedora, Red Hat, and Ubuntu), Windows (Vista and XP), and Mac systems
are all widely used.  DCO also provides e-mail service, and many
faculty members and EECS graduate students use this as their primary
e-mail address.


\item CAEN workstations\\
\label{sec-7_1_4_2}%
CAEN maintains several hundred workstations running Linux and Windows
in a dual-boot configuration.  All these machines all share a common
distributed file system.  Most classes that involve programming use
these machines.  They are found in “workstation clusters” in all of
the Engineering buildings.  There are two labs of these machines on
the ground floor of the CSE building.


\item Macintosh workstations\\
\label{sec-7_1_4_3}%
There are several high-end Mac clusters around campus.  The largest of
these is located on the 2nd and 3rd floors of the Duderstadt Center.
They are loaded with a variety of software including word processors,
graphics and spreadsheet programs, and network interfaces.  DCO
maintains a small Mac cluster in the west CAEN lab in the CSE
building.


\item Wireless\\
\label{sec-7_1_4_4}%
Wireless network coverage is available in almost all engineering
buildings, including large portions of the EECS building, CSE Building
and the Duderstadt Center.  To access the network, you may need to
download and install the Cisco VPN (Virtual Private Network) client
software, and you need to authenticate using your CAEN or ITCS
account.  More details are available at
\href{http://wireless.engin.umich.edu/}{http://wireless.engin.umich.edu/} along with downloads of the necessary
VPN client.
\end{itemize} % ends low level
\subsubsection{Accounts and Email}
\label{sec-7_1_5}

CAEN, DCO, and ITCS each provide computer accounts to students.  All
accounts provide remote login and file space and can be used for your
work.  Additionally, DCO and ITCS provide e-mail and space for a
personal web site.  Using the campus network, you can use any of your
accounts to access the files on your other accounts.
\begin{itemize}

\item CAEN accounts\\
\label{sec-7_1_5_1}%
Once you have your uniqname, you can fill out a form to obtain a CAEN
account.  This will allow you to use any of the CAEN machines.  Your
uniqname and CAEN password will also let you log into the personal
computers maintained by ITCS (most of these are located on Central
Campus).


\item DCO accounts\\
\label{sec-7_1_5_2}%
Once you have obtained a uniqname, you can register your DCO account.  Pick up an application at 3909 CSE, or when you meet with Dawn.  You need Dawn’s signature to complete this application.  This account (which will also use your uniqname as the login name) will allow you to access the DCO public machines.  If you are part of a departmental research laboratory, you will be able to use your DCO username on the workstations in that laboratory as well.
Your DCO e-mail address will have the form:  \texttt{uniqname@eecs.umich.edu}.
If you want your e-mail forwarded to another account, send a request
to help@eecs.umich.edu.


\item ITCS accounts\\
\label{sec-7_1_5_3}%
Your ITCS account will give you access to the computers in the ITCS
public computer labs. You can also use your CAEN password at these
computers.  Simply use \texttt{uniqname@engin.umich.edu} and your CAEN
password to login.  The e-mail address of your ITCS account is
uniqname@umich.edu.  You can forward your e-mail to another place by
modifying your entry in the X.500 server (see next section).
\end{itemize} % ends low level
\subsubsection{Campus-wid X.500 directory}
\label{sec-7_1_6}

The University maintains an X.500 directory service which contains information about all University students and employees. You can modify your directory listing to tell people how to contact you.
The directory allows everyone to have an e-mail address alias of the form uniqname@umich.edu. Your X.500 entry determines to which of your computer accounts this e-mail address alias points.  You can access X.500 at \href{http://directory.umich.edu/}{http://directory.umich.edu/}.
The Office of the Registrar updates the name, title, address, and phone fields in your X.500 entry each month. If you do not want them to do this, you can set the “batch updates” field to OFF in your entry.
You can view your X.500 information (or anyone else's) by typing “finger name@umich.edu” at any Unix prompt.  The name can be either a uniqname or part or all of the full name of the person whose entry you wish to view.
By changing your X.500 entry, you can also change where mail is
delivered when it is sent to your main university account:
\texttt{uniqname@umich.edu}.
\subsubsection{Connecting from home}
\label{sec-7_1_7}

ITCS provides the BlueDisc software bundle to help home computers connect to campus networks (available at \href{http://www.itd.umich.edu/bluedisc/}{http://www.itd.umich.edu/bluedisc/}.
You have two basic options for broadband connectivity off campus.
Comcast offers cable modem service for around \$60 per month.  Download
speeds are approximately 6 Mbps and upload speeds around 350 Kbps.  As
an alternative, AT\&T offers DSL service.  DSL availability and speeds
are subject to where you live, and if available, DSL costs around \$60
per month.  Advertised speeds are 6 Mbps download and 768 Kbps upload
(but don’t be surprised if it’s less).
\subsubsection{Printing}
\label{sec-7_1_8}

DCO maintains several public printers around CSE.  You can get a list of them at: \href{http://www.eecs.umich.edu/dco/files/drivers/drivers.php}{http://www.eecs.umich.edu/dco/files/drivers/drivers.php}.
CAEN provides laser printers in its computer labs, located in most engineering buildings (but not Duderstadt Center).  You start with an allocation of \$60 towards printing.  B\&W pages are 3 cents apiece; color pages are 12 cents.  Once that allotment is used, you pay for your own prints.
ITCS manages printers in the Duderstadt Center and other computer
labs.  You can print 700 B\&W pages before being charged 6 cents/page.
Color prints always cost 40 cents.
\subsubsection{How to get help}
\label{sec-7_1_9}
\begin{itemize}

\item Help from DCO\\
\label{sec-7_1_9_1}%
E-mail: help@eecs.umich.edu
Web:    \href{http://www.eecs.umich.edu/dco/contact.php}{http://www.eecs.umich.edu/dco/contact.php}
Office: 2917 CSE
DCO is helpful at supporting nonstandard, customized Sun hardware configurations and Linux (especially Fedora, RedHat, and Ubuntu).
6.8.2 Help from CAEN
Phone Hotline:  763-5041 between 8:00 A.M. and 5:00 P.M.
Walk in:        1315 Duderstadt Center between 8:00 A.M. and 5:00 P.M.
On-line:        Run the caenhelp program or on the web \href{http://www.engin.umich.edu/caen/}{http://www.engin.umich.edu/caen/}.
Classes:        CAEN offers computer classes at the beginning of each semester. Schedules are posted outside the CAEN labs.
Handbook:       The CAEN handbook is available at 1315 Duderstadt Center and at CAEN labs.
Newsletter:     CAEN publishes a monthly newsletter. You can get a copy in the CAEN labs or in the EECS mail room.
Tip Sheets:     Available outside 1315 Duderstadt Center.

\item Help from ITCS\\
\label{sec-7_1_9_2}%
Phone:  ITCS's help line operates 24 hours a day.  Dial 764-HELP.
E-mail  online.consulting@umich.edu
Walk in:        The larger ITCS sites (e.g. Angell Hall) have counselors on duty during the day.
Web:    \href{http://www.itcs.umich.edu/}{http://www.itcs.umich.edu/}
Newsletter:     Every month ITCS publishes Information Technology Digest. Copies are available at ITCS labs and in the EECS mail room.x
\end{itemize} % ends low level
\section{Bookstores and Libraries}
\label{sec-8}
\subsection{Libraries}
\label{sec-8_1}

The University of Michigan Libraries are spread all over campus; the
two branches you will probably use most are the Duderstadt Center and
the Graduate Library.  The Duderstadt Center contains the entire
collection of engineering-related books and periodicals in a special
compact shelving system.  The Harlan Hatcher Graduate Library is the
hub of the university library system, containing books from almost
every academic discipline.

If you are a GSI or GSRA, you are entitled to have a “GEO account” in
addition to your “student account.”  Books may be checked out under
either account using your Mcard, but books checked using the GEO
account usually have a longer due date, and you will not be charged a
fee if you return the books late.

If you live in Ann Arbor, you can also get a free library card to use
at the Ann Arbor public libraries.  The main branch is at the corner
of Fifth Ave. and East Williams downtown (343 Fifth Ave).  The nearest
branch to campus is the Traverwood branch, found by heading north on
Huron Pkwy to Traverwood (3333 Traverwood Dr).  In addition to the
usual books and periodicals, branches have videos, tapes, compact
discs, and DVDs available for checkout.
\subsection{Textbooks}
\label{sec-8_2}

Your staff ID card (for GSIs and GSRAs) may get you a discount from some of the bookstores that sell textbooks.  Several of the stores that used to do this have stopped, so ask at each store.  GEO (995-0221) has a list of local stores that offer discounts to those with ID cards.  There are three major stores that sell textbooks in Ann Arbor.
\subsubsection{Barnes \& Noble}
\label{sec-8_2_1}

There are two of these: in the Pierpont Commons and in the basement of the Michigan Union.  The Commons store is fairly small, but carries most of the textbooks used in classes on North Campus (Engineering, Computer Science, Art, Architecture, Music).  It also sells newspapers, art supplies, and some snack foods. The store in the Union is much bigger, and carries textbooks for all of the courses offered at the University.
\subsubsection{Ulrich’s Bookstore and Electronics}
\label{sec-8_2_2}

Located at the corner of E. University and S. University (549 E. University Ave., to the southeast of Central Campus), Ulrich’s has textbooks for all University courses (including Computer Science). It also carries a large selection of souvenir clothing, school and art supplies, and general-interest books. There is an annex next door that sells computers, software, and calculators.
\subsubsection{Michigan Book and Supply}
\label{sec-8_2_3}

This store carries much the same selection of textbooks and merchandise as the other two, but offers more clothing and art supplies and fewer general-interest books. Michigan Book and Supply is located at the corner of State St. and North University (317 S. State St.)
All three stores buy and sell used textbooks, with special buy-back drives at the end of each term. If you have time, it pays to shop around for used books. One more warning: try not to buy books during the first week of classes – it takes hours to get through the long lines.
\subsubsection{Student Book Exchange}
\label{sec-8_2_4}

This is a student organization that runs a used book exchange during 2 days at the beginning of each term. Prices for buying and selling books are usually better than the bookstores. Look for flyers announcing the time and location of the book exchange when you arrive on campus. Go early and expect long lines.
\subsection{Course Packs}
\label{sec-8_3}
\subsubsection{Dollar Bill Copying}
\label{sec-8_3_1}

Expect very long lines. Go early in the morning or after the first few days of class. (611 Church Street)
\subsection{Book dealers of new books}
\label{sec-8_4}

Barnes \& Noble
This is the other large bookstore in Ann Arbor. They also sell
software. (3245 Washtenaw at Huron Parkway)
\subsection{Book dealers (used)}
\label{sec-8_5}

If you're looking for used books, here are some good places to start:
\subsubsection{Afterwords}
\label{sec-8_5_1}

Located at 219 S. Main Street.  Most of their books are overruns and returns.
\subsubsection{Books in General}
\label{sec-8_5_2}

Books in General is one of the few used bookstores in town that tries to get technical books. Occasionally they'll also get special shipments of like-new books, which sell for 60-75\% of retail. (332 S. State Street)
\subsubsection{David’s Used Books}
\label{sec-8_5_3}

Located next to Border’s on E. Liberty at S. State, David’s has a large selection of used paperback and hardcover books of all genres.
\subsubsection{Dawn Treader}
\label{sec-8_5_4}

The biggest used bookstore in Ann Arbor (514 E. Liberty Street).  Good selection here too.
\subsubsection{Friends of the Public Library}
\label{sec-8_5_5}

Located in the basement of the public library, at 343 Fifth Ave.  Only open on Saturday and Sunday, from 1:30 to 4:00, and only during 9 months of the year.
\section{Study Spaces}
\label{sec-9}

Quiet space in which to study is a rare commodity on campus. Here are some of the favorite places of EECS students:

\begin{itemize}
\item The Blue Lounge in the G. G. Brown building (connected to EECS) is pretty quiet and has vending machines.
\item The Duderstadt Center has a number of study rooms on the second floor, as well as plenty of tables in the stacks for quiet study.  Tends to be packed during the last weeks of class.
\item In the central campus area, the School of Public Health and the public health library are usually quiet and empty.
\item If you really need absolute quiet, try the Reading Room or the carrels in the Harlan Hatcher Graduate Library (3rd – 6th floors) or the Reading Room in the Law Library on Central Campus.
\item Tishman Hall in the CSE building is a large open atrium and a
  popular place to study.  There are also many conference rooms
  scattered throughout the building that are usually available.
\end{itemize}
\section{Housing}
\label{sec-10}

Ann Arbor offers a large variety of accommodations.  The Housing office (+1 734-763-3164) in the Student Activities Building (SAB) provides detailed information on both University-owned and private housing, including maps, bulletin boards and listings.  This office is also a good place to find advertisements for roommates and sublets.  The University Housing office web page is \href{http://www.housing.umich.edu/}{http://www.housing.umich.edu/}.
To help incoming graduate students find apartment-mates, a mailing list has been set up so that you can communicate with other incoming graduate students.  To join the housing mailing list (cseg-apts@eecs.umich.edu), please send e-mail to cseg-apts-request@eecs.umich.edu with the subject “subscribe”.
\subsection{Location}
\label{sec-10_1}

Many students feel that North Campus is too far away from the center of campus social life.  Others like the peace and quiet of North Campus.  The area between Kerrytown and the Hospital is recommended as a good balance between these.  Housing is expensive close to either campus.
Places outside Ann Arbor offer better prices than those within the city.  The apartment complexes in the neighboring city of Ypsilanti are a convenient 10 - 25 minute drive to campus.  While a typical two-bedroom apartment within walking distance of campus might cost more than \$1000 per month, an equal-sized apartment in the Barrington Heights complex in Ypsilanti is about \$800.
\subsubsection{Residence Halls}
\label{sec-10_1_1}

University residence halls have both single and double rooms, but are small and cramped. Several different meal plans are available. The Vera Baits complex is a 10 minute walk from EECS, and there are other residence halls in the Central Campus area. The University's rent is high compared to a shared two-bedroom apartment. Rooms in the residence halls are connected to the campus network via Ethernet.
\subsubsection{Cooperative Living}
\label{sec-10_1_2}

The Inter-Cooperative Council is a federation of over a dozen cooperative living houses.  The North Campus Cooperative consists of Renaissance and O’Keeffe Coops.  They are a 10-minute walk from EECS.  The residents are primarily graduate students from a variety of disciplines, but there are some undergraduates.  In addition, there are many international students and non-students.  Dinner is served every night and there is brunch on the weekends.  The amenities include TVs, VCRs, a pool table, a music room with a piano, Ping-Pong table, and computer room.  During the school year rent is under \$550/month for small, furnished singles.  Over the spring and summer the cost is much less.  Housing, food, electricity, water and usage of all recreational facilities are included in the rent.  Residents are required to contribute 4 hours of labor (cooking and cleaning) each week. Call 734-662-4414 for more information.
\subsubsection{University Family Housing}
\label{sec-10_1_3}

Housing a family can be quite expensive in Ann Arbor.  A comparatively cheap alternative is the University's family housing complexes.  There are about 1700 units, ranging in size from efficiencies to 3 bedrooms, both furnished and unfurnished.  In general, the larger units are reserved for students with families.  When there are vacancies, single students may apply for smaller units, but they have a lower priority than families.  Most of the family housing is in the North Campus area.  The Northwood I, II and III complexes containing apartments, and Northwood IV and V have townhouses.  Northwood I-IV are within 5 to 10 minutes walking distance of EECS, and Northwood V is served by a free shuttle bus.
\subsubsection{Apartments}
\label{sec-10_1_4}

There are several large apartment complexes near North Campus.  Most
of them are unfurnished 1 and 2 bedroom apartments.  Three bedroom
apartments are rare, with a few in Huron Towers, Traver Ridge, and
Island Drive.  Prices range from about \$600-700 for efficiencies to
\$800-1100 for 2 bedroom apartments. All of the apartment complexes
near North Campus have adequate parking space.

Most of these places give a \$50-\$100 bonus to current residents who
recruit new residents. You can probably find someone living there with
whom you can split the bonus. Shop around for good deals.

Also, make sure to check out the Homestead Tax credit in the Michigan
income tax forms.  It can give you a substantial refund (hundreds of
dollars) based on the amount of rent you pay and your income.

Here are some evaluations of local apartment complexes. Many have a
pool, tennis courts, and common rooms. The list is not intended to be
comprehensive:


\begin{itemize}
\item The Highlands on Broadway St. is a10 minute walk from CSE.
  Apartments have dishwashers.  Laundry is available in each building.
  You get a parking sticker for each bedroom in your apartment (1
  or 2) and there are usually extra available.  There is a small pool
  and picnic area. +1 734 769-3672.
\item The Willowtree apartments on Plymouth Road are about a 10 minute
  walk from the CSE building. Close to grocery store.  Swimming pool
  and tennis courts. +1 734 769-1313.
\item The Parc Pointe apartments are also very close to CSE.  They are
  very nice, but expensive.  Each apartment has laundry hookups for
  your own washer and dryer.  There is also a separate laundry
  building.  Call +1 734 769-1450 for leasing.
\item Traver Ridge is on Plymouth Rd. about a 15 minute walk from
  campus. The apartments are nice and the rent is reasonable.
\item Broadview apartments are near north campus, next to the Highlands.
  It is in a convenient location, within 10 minute walk from North
  Campus and across the street from Wendy’s and Subway.
\item Greenbrier is a little further from campus, about 1.5 miles.
  Olympic size swimming pool.  Exercise room with treadmills and
  weights. +1 734 665-3653.
\item Huron River Plaza is about ten minutes from CSE. +1 734 996-4992.
\item Huron Towers is close to campus, next to the Arboritum and Mitchelle Field. +1 734 665-9161.
\item The Island Drive Apartments are about a 25 minute walk from CSE.
  Their location is scenic (near a river and park), and they are very
  quiet. +1 734 665-4331.
\item Medical Center Court Apartments is next to Island Drive Apartments.
\item The Courtyards is located on north campus near CSE. +1 734 994 6007
\end{itemize}

Closer to downtown, you can find houses that have been split up into
apartments, or whole houses that are rented to groups of students.
These get mixed reviews; depending upon the landlord, the experience
can either be a blast or a disaster.  There are also regular apartment
complexes near Central Campus, and they usually have adequate parking
space.  Most apartments in the Central Campus area are furnished, and
tend to be much more expensive than comparable apartments near the
North Campus.
\subsubsection{Other Options}
\label{sec-10_1_5}

There are many small houses and condos that even graduate students may
be able to afford.  Rent in Ann Arbor can be more expensive than the
corresponding mortgage payments.  If you plan to be in Ann Arbor for a
while, you may actually save money by not having an apartment.  Just
be sure you can accept the inherent risks associated with owning and
reselling property.
\subsubsection{Advice}
\label{sec-10_1_6}

It's very expensive to live by yourself in Ann Arbor: expect to pay at
least \$700 for a one-bedroom.  Unless you are in a big complex, pay
careful attention to parking.
\section{Transportation}
\label{sec-11}

Getting to and from the University is one of the first challenges that
you will face.  This task can be complicated by the fact that there
are two campuses, a little over a mile apart, but most of your classes
will be on North Campus.  You will also have to travel the Central
Campus in order to visit the main libraries, University offices, and
most of the bookstores, restaurants, and bars.  If you choose to live
near Central Campus, you will be closer to these things but will have
to commute to North Campus every day.

Parking and Transportation Services (PTS) handle most
transportation-related issues.  The PTS home page is located at
\href{http://www.pts.umich.edu/}{http://www.pts.umich.edu/}.
\subsection{University Bus}
\label{sec-11_1}

Transportation between the campuses is provided free by the
University.  Blue shuttle buses run every 5-10 minutes from the
C.C. Little building on Central Campus to the Pierpont Commons.  From
there, one set of buses goes up the hill to the Bursley-Baits
dormitory complex, while the others travel along Bonisteel Blvd. and
up the hill behind EECS toward the Northwood housing units and the
commuter lots.  The best place to wait for a bus to Central Campus is
the corner of Bonisteel and Murfin, right outside the Pierpont
Commons.

If you travel to Central Campus by shuttle bus, you should get off at
the main stop by the C. C. Little building (you can recognize it
because there are permanent bus-stop shelters.)  From here you can
walk west down N. University Ave. to State street, and then turn left
to get to the Union and the administration buildings.  To get to the
libraries, walk between Chemistry and Natural Resources and cross the
“Diag”.

Bus maps are available on board buses as well as at the Campus
Information Center in the Pierpont Commons.  The University buses run
every 20-30 minutes at night and on weekends.  You can find the
approximate position of busses and wait times for most routes on
\href{http://mbus.pts.umich.edu/}{http://mbus.pts.umich.edu/}.
\subsection{City Bus}
\label{sec-11_2}

The AATA (Ann Arbor Transit Authority) runs the city buses.  City
buses run about every half hour, and travel from the main terminal
downtown out to the various parts of the city.  One route goes through
Central Campus, and another (the \#3) goes through North Campus.  Rides
are free with your Mcard; otherwise full fare is \$1.25, and a monthly
pass is \$48.  Schedules and information may be obtained at the
terminal at Fourth Avenue and Williams, downtown, or online at
\href{http://theride.org/}{http://theride.org/}.
\subsection{Biking}
\label{sec-11_3}

This is a very popular option.  Ann Arbor is a relatively
bicycle-friendly city.  Many of the major streets have bike paths or
bike lanes.  There is a bike path between the two campuses, running
along Fuller road.

Even though the winters are fairly cold, there are only a few days
each year when the roads and paths are too icy or snowy for safe
biking.  If you plan on riding a bike through the winter, it's good to
get knobby tires and to clean and lubricate your bike frequently
against damage from road salt.  Lock your bike with a good U-lock,
because thefts are prevalent.  It is also recommended that you wear a
helmet at all times, and reflective clothing at night.

If you don't have a bicycle, there are several stores selling both new
and used bicycles and equipment, and Craigslist is another useful
option.  Your best source of information is to ask around the
department to find out which shops people prefer.  Consider buying a
mountain bike or “hybrid” instead of a road bike.  This will give you
an advantage on hills and in snow and mud.
\subsection{Cars}
\label{sec-11_4}

If you have a car, or choose to buy one here, your main problem will be on-campus parking.  Soon after your arrival and after you have obtained a student ID, you should visit the PTS office at 777 North University (764-8291).  They will give you a parking map and a rule book.  When you go to PTS, be sure to take the following with you:

\begin{enumerate}
\item Student ID Card
\item License Plate Number(s)
\item Driver's License
\item Vehicle Registration(s)
\end{enumerate}
\subsubsection{Parking Options}
\label{sec-11_4_1}

For parking on North Campus, if you desire to be a law abiding
commuter, there are basically three options:

\begin{itemize}
\item Buy a student parking permit, which permits you to park in special student lots.
\item Feed the parking meters at metered parking spots.
\item Use the park-and-ride lot on Green Rd (Lot NC37).  Parking is free without permit, but overnight parking is not allowed.  The AATA bus will take you to north campus for free if you show your student ID.
\end{itemize}

Refer to the PTS web site for parking lot maps.
Warning: The city and University aggressively enforce parking
regulations. If you park illegally, you risk getting a parking
ticket. Be aware that parking tickets are \$20 for parking illegally in
the blue lots, and \$10 for an expired meter.
\begin{itemize}

\item Student Parking Permits\\
\label{sec-11_4_1_1}%
A Student Orange Permit allows parking in any Orange lot. The North
Campus Orange lots are NC7, NC46, NC51, NC52, and NC78.  The price is
\$70 (rates given for 2009-2010) and can be prorated based on when
during the year you purchase your pass.  These are available anytime,
but there are only a fixed number of permits available.  A Student
Yellow permit is valid in NC40, or any Yellow lot – except on the
Medical Campus, and is only available to graduate students.  This
means that the lot, at least on North Campus, is less likely (than
Orange) to fill up. The North Campus Yellow lots are NC9, NC34, NC53,
NC61, NC62, and NC63.  This permit is also honored at any Orange lot
except for NC40.  Permits are available at the Parking Services office
for Graduate Students. Cost is \$141.

All parking permits can be purchased online through the main PTS web
site


\item After Hours Permits\\
\label{sec-11_4_1_2}%
The After Hours Permit allows parking in any Blue parking area after 3
P.M. Monday through Friday and throughout the weekend.  Permits are
available at the Parking Services office for Graduate Students only.
The annual cost is \$54.  This permit can also be used in conjunction
with an Orange or Yellow permit.  A Student Orange permit with an
After Hours decal is \$124.  A Student Yellow permit with an After
Hours decal is \$195 (and again all prices can be prorated).


\item Non-permit parking (metered spots)\\
\label{sec-11_4_1_3}%
If you don't drive to campus all that often and don't mind feeding
some meters, there are metered spaces at Pierpont Commons, behind the
ATL, on top of the hill behind the Naval Architecture building, and
just outside CSE.  The meter rates are around \$1 per hour.


\item Free After Hours Parking\\
\label{sec-11_4_1_4}%
The G.G. Brown parking lot and auxiliary lot are open for unrestricted
parking after 5:00 P.M.  Other lots on North Campus parking lots are
free after 6:00 P.M.  Except for the last week of the term you should
have no trouble finding a spot.


\item Central Campus Parking\\
\label{sec-11_4_1_5}%
The faculty/staff garages are sometimes free after 6:00 P.M., but some
are restricted until after 10:00 P.M.  Be sure to check the hours
before entering — they seem to change with dizzying frequency!  In
some of the neighborhoods around campus (such as south of Hill on
Church \& Forest) you can find on-street parking, but you have to be
willing to look around.  If you park at a metered spot, feed the meter
diligently. The city derives a lot of revenue from parking tickets!
An After Hours permit can be very helpful for parking on Central
campus after 3 P.M.


\item Registering your vehicle\\
\label{sec-11_4_1_6}%
If you are making this state your permanent residence, you can
register your car and get a Michigan driver's license at the local
branch office of the Secretary of State (2121 West Stadium Blvd.,
665-0627).  This is located on Stadium Blvd., just south of Liberty —
about three miles west of Central Campus.  If you are already a
resident of Michigan from another part of the state, it's probably a
good idea to inform them of your move and get an updated driver's
license.

\end{itemize} % ends low level
\section{Activities}
\label{sec-12}

When you've had enough of school and paperwork, there is plenty to do
in Ann Arbor to take your mind off of your academic career.  Here is a
sampling of the favorite pastimes of some current EECS students:
\subsection{Friends}
\label{sec-12_1}

One of the easiest ways to meet new people is to start talking to the
other students in your classes.  Many of them will also be new
students who are similarly interested in meeting people.  Many of the
first year classes require group projects, which give you an
opportunity to spend many hours with your classmates.  The weekly CSEG
happy hours are a good place to meet new and old graduate students, as
well as a professor or two.

Rackham also hosts an interdepartmental happy hour on Friday
afternoons. It is a good way to meet students from outside of EECS.
For more information check out
\href{http://www.rackham.umich.edu/student_life/}{http://www.rackham.umich.edu/student\_life/} (which also lists other
Rackham social events) or social$_{\mathrm{grads}}$@umich.edu.

Roommates are a good way to broaden your horizons.  Room with someone
outside the department or choose cooperative housing.  There are
usually many advertisements from people looking for roommates on the
cseg-apts email list or posted on the bulletin boards in the EECS
building.

How about table tennis or folk dancing?  Religious groups, political
  groups (including the GSI union), sports clubs, art and music societies, clubs for students from different countries, and dozens
  of other special interest groups provide places to interact with
  people both within and completely outside of your academic world.

  There is sure to be one that matches your interests.  For a listing
  of all student group activities on a given day, check out the U-M
  Calendar on page 3 of The Michigan Daily.
\subsection{Fun and Games}
\label{sec-12_2}

There is plenty to do in Ann Arbor, most of it inexpensive.  The Ann
Arbor Observer (a local news magazine) has monthly entertainment
listings, and AnnArbor.com and The Michigan Daily both have
entertainment sections near the end of each week. Below are several
ideas that we have come up with.
\subsection{Athletics}
\label{sec-12_3}

CSEG fields teams in several of the intramural leagues, and there are
often pickup games of volleyball, ultimate frisbee, etc. on the green
in front of EECS. The North Campus Recreation Building, up the hill
from the Pierpont Commons at the corner of Hubbard and Murfin, offers
basketball, volleyball, and badminton courts, a weight room, and a
pool. They also rent sporting and camping equipment to students. If
you want to get a locker there, you have to show up at about 6:00
A.M. on the announced day and stand in line for several hours.
\subsection{Parks and Gardens}
\label{sec-12_4}

Matthaei Botanical Gardens
Arboretum (near the Medical campus along Huron river, great place for a romantic walk)
Gallup Park (good place to try out roller blading)
Huron River (Canoe from Argo Livery to Gallup Park)
\subsection{Concerts and Performances}
\label{sec-12_5}

There are many student performance groups on campus, many of which are
open to grad student performers as well as undergrads.  Some of the
more popular shows include the Halloween Concert given by the
Orchestra and the Monsters of A Cappella put on by several informal
singing groups.

Besides the usual concerts and performances by student bands,
orchestras, theater and dance companies, the University owns two large
auditoria which book touring performances.  These are Hill Auditorium
and the Power Center.

There are also several performance venues in Ann Arbor which are run
by non-profit community organizations.  The Michigan Theater, restored
to the glory of its heyday, shows classic movies, art films, and live
acts of various kinds.  Performance Network (408 W. Washington St.)
shows out-of-the-ordinary productions, by several local theater
companies.  One of Ann Arbor's best kept secrets is The Ark (637
S. Main St. between Liberty and Williams).  Here you can find live
folk music, jazz, progressive, world-beat, and much more.  You can
volunteer to usher at any of the places mentioned in this section,
which often gets you in for free.

The University Musical Society has a half-price ticket sale for
students at the beginning of each semester.  This is a great way to
get cheap tickets to the big name shows that come through Ann Arbor.

The new Walgreen Drama Center and Arthur Miller Theatre recently
opened on north campus next to the CSE building.  Over the next year,
the Walgreen Center will host performances by the Departments of
Theatre \& Drama and Musical Theatre.
\subsection{College Radio}
\label{sec-12_6}

Ann Arbor is lucky to have one of the more prominent freeform college
radio stations left in the nation, WCBN FM, 88.3.  They play a variety
of freeform, electronic, and jazz shows during the work week, with
public affairs shows in the early evening.  The weekends feature a
variety of specialty shows, ranging from folk and blues to African and
Indonesian music.  You can listen on your radio or on the web at
\href{http://www.wcbn.org/}{http://www.wcbn.org/}.  WCBN is a student-run organization and is a
great group to meet new people, both students and non-students alike,
as well as a chance to be exposed to a lot of new music.  Email
training@wcbn.org if you’re interested in being involved.

Other student favorites include WUOM FM 91.7 (NPR,
\href{http://www.michiganradio.org/}{http://www.michiganradio.org/}) and WEMU FM 89.1 (Jazz/NPR,
\href{http://www.wemu.org/}{http://www.wemu.org/}).
\subsection{Movies}
\label{sec-12_7}

Michigan Theater (foreign, esoteric, film festivals, classics)
State Theater (independent, cult classics)
Quality 16 (first run)
Showcase Cinema (first run)
Film societies on campus (cheap or free, usually at Angell Hall or the MLB on Central Campus)
Ann Arbor Summer Festival (“Top of the Park”) - Free outdoor movies in June and early July
Dollar Movies at Briarwood (second run, \$1)
Compuware Arena Drive-In Theaters (first-run)
\subsection{Farmers' Market}
\label{sec-12_8}

Every Wednesday and Saturday, local farmers gather at the municipal market at Catherine and Fifth Avenue.  You can find good fresh fruit, vegetables, baked goods and plants.
\subsection{Bars}
\label{sec-12_9}

Arbor Brewing Company (brew-pub)

Alley Bar (martini bar)

Ashley's (huge beer selection)

Blind Pig (live music, cover)

Casey’s (townie)

Connor O'Neal's (Irish themed, cheesy)

Dominick’s (The original CSEG hangout.  Beer served in canning jars.
Restaurant)

Full Moon Bar (lots of pool tables)

Good Time Charlie's (collegey, late-night drink specials, restaurant)

Grizzly Peak Brewing Co. (brew-pub, restaurant – Great burgers)

Mitch’s (also collegey, but lots of cheap beer, pitcher specials)

Old Town (townie)
\subsection{Places to Eat}
\label{sec-12_10}

Ann Arbor has approximately 200 restaurants that provide a wide
variety of American and ethnic specialties ranging in price from \$5 to
\$50 or more for a meal.  For a complete listing of restaurants in the
area see the phone book or the Discover the Greater Ann Arbor Area
guide published by The Ann Arbor News. The Ann Arbor Observer and the
City Guide also have good restaurant listings.
\subsection{Other Sources of Information}
\label{sec-12_11}

This guide is not big enough to hold all of the information that you will need.  Here are some places you can look for more:
\subsubsection{Fellow graduate students}
\label{sec-12_11_1}

Ask your fellow graduate students for recommendations.
\subsubsection{University sources}
\label{sec-12_11_2}

A good place to start is the Campus Information Center.  The main
center is on the second floor of the Michigan Union.  A smaller one is
at the Pierpont Commons.  They can be reached at 763-INFO, and are
usually able to answer even your strangest questions.

Pick up a copy of Rounding out A2 (in case you don't already have it).
Copies are usually available at Rackham (the Graduate School office)
and at the International Center.  It will also be available in 3310
EECS after classes begin.  This is a very detailed book with lots of
information about Ann Arbor and is a very useful reference.
\subsubsection{Local publications}
\label{sec-12_11_3}

AnnArbor.com is an online-only newspaper that recently replaced The Ann Arbor News, leaving Ann Arbor without a mainstream print newspaper.
Current is a complimentary entertainment monthly, with information on
concerts, cinema, etc.

The Ann Arbor Observer is a monthly publication.  It has a detailed
restaurant section, and lists shows and events in the surrounding
area.  In addition, it is a good source of information about local
history, politics, and personalities.  The City Guide edition is
published in the fall.  It has excellent information about living in
Ann Arbor and includes maps, listings of shops, restaurants, and
things to do.

The Michigan Daily is the student newspaper, and has listings of various campus events.
The Metro Times is a weekly newspaper that lists entertainment events in Detroit.
\href{http://www.mlive.com/ann-arbor/}{http://www.mlive.com/ann-arbor/} lists restaurants and entertainment events in Ann Arbor.
An excellent resource for all things Ann Arbor is the ArborWiki, located at \href{http://arborwiki.org/}{http://arborwiki.org/}.
11.4.1  International Students
The International Center (603 E. Madison St., +1 734 764-9310), adjoining the south side of the Michigan Union on Central Campus, is a wonderful resource.  This should be one of your first stops as soon as you get to campus. There are several international student organizations.  The International Center can put you in contact with them.
11.5 Libraries and Museums
Visit the North Campus Engineering Library in the Duderstadt Center.
Visit the Harlan Hatcher Graduate Library and the Shapiro Library (“UgLi”) on Central Campus.  You can take the shuttle bus to get there.  While on the bus, pick up a free bus schedule.
Check out the art museum across from the Michigan Union, and the museum of Natural History by the Central Campus shuttle bus stop.
\section{How to Get to Campus}
\label{sec-13}

By air:
Fly into Detroit Metro Airport (airport code: DTW).  NOTE:  do not fly into Detroit city airport, since there is no convenient transportation from there.  Here are some of the shuttle services available:

Michigan Flyer
\href{http://www.michiganflyer.com/}{http://www.michiganflyer.com/}.
(517) 333-0400
Bus runs every couple hours, cost \$15 per-person one-way.
Must make reservations at least one hour before departure, through web site or over phone.


Ann Arbor Airport Shuttle Express\_{}
\href{http://www.annarborairportshuttle.net/1.html}{http://www.annarborairportshuttle.net/1.html}
(734) 394-1665
reservations@annarborairportshuttle.net
Runs 6:00 am to 9-9:30 pm, cost \$35 per-person one-way, \$60 round trip.
Must call ahead for reservation


Custom Transit, Inc.
\href{http://www.customtransit.com/}{http://www.customtransit.com/}
(734) 971-5555
Runs 4:30 am to 11:00 pm, cost \$30 per-person one-way.
Additional cost for pick-ups between 12:30 am and 4:30 am –contact them
Must call ahead for reservation


Metro Eddie’s Super Shuttle
\href{http://annarborsupershuttle.com/}{http://annarborsupershuttle.com/}
(734) 507-9220
Runs 7 am to 7 pm, cost \$40 per-person one-way.
Must call ahead for reservation


Accent Transportation Service
\href{http://www.atsride.com/}{http://www.atsride.com/}
(800) 346-9884
Runs 24/7, cost \$38–48 per-person one-way.
Must call ahead for reservation

More information from the Ann Arbor Visitor’s Bureau:  \href{http://www.visitannarbor.org/}{http://www.visitannarbor.org/}.
\textbf{By car}:
From the south or east, go to Toledo and take US 23 north to I-94.  Go
west on I-94, exit State St.

From the west, take I-94 east from Chicago.  Take exit number 177,
State St.

Go north on State St. for 3 miles to Central Campus.  You will see the
athletic complex on your left.  Then cross Hill street, and look for a
set of lovely gothic buildings on your right.  This is the Law school.
In the next block on your left is the Michigan Union, a large brick
building with a big tower in the middle of the facade.  There is a
campus information booth in the Union.

The EECS department is on North Campus, which is about a mile and a
half to the northeast.  Continue going north on State Street.  Turn
right onto Ann Street.  At the traffic light, turn left onto Glen
Street.  Glen Street becomes Fuller Road and passes the Hospital.  You
will pass the Huron River and some athletic fields.  At the traffic
light, turn left onto Bonisteel Boulevard.  You will see a large sign
for North Campus.  Drive up Bonisteel. The Pierpont Commons is at the
intersection of Bonisteel and Murfin.  The EECS building is across the
green behind the Pierpont Commons.

You can also get to North Campus from US-23 by taking the Plymouth
Rd. exit.  From US 23 going north, take a left onto Plymouth Rd.  Go
through several lights until you hit Murfin/Upland.  Take a left, and
you’re on North Campus.  This road runs right into Pierpoint Commons
and Bonisteel Rd.

To get back to main campus, just take Bonisteel down the hill from the
Pierpont Commons and turn right onto Fuller.  As it passes the
hospital, the road becomes Glen.  Take Glen until it ends at
Washtenaw, and turn right.  Then turn immediately left onto Fletcher.

\textbf{On campus}:
The University operates free shuttle buses which run every ten minutes
between Central Campus and North Campus.  The buses are blue and have
the word “Michigan” painted in gold on the sides.

These buses travel along Fuller road between the campuses, and stop at
the Pierpont Commons before continuing on to the North Campus
dormitories and housing complexes.
\section{Acronyms}
\label{sec-14}


\begin{center}
\begin{tabular}{ll}
\hline
\hline
 \textbf{Acronym}  &  \textbf{Meaning}                                                                    \\
\hline
 ACAL              &  Advanced Computer Architecture Lab.  Part of CSE.                                   \\
 ACM               &  Association for Computing Machinery.  Professional society                          \\
 AI                &  Artificial Intelligence                                                             \\
 ATL               &  Advanced Technologies Lab.  Home of artificial intelligence.                        \\
 CAEN              &  Computer Aided Engineering Network.  Computing services for School of Engineering.  \\
 CSE               &  Computer Science and Engineering.  A division of EECS.                              \\
 DCO               &  Departmental Computing Organization.  Computing services for EECS.                  \\
 ECE               &  Electrical and Computer Engineering.  A division of EECS.                           \\
 EECS              &  Electrical Engineering and Computer Science Department                              \\
 EES               &  Electrical Engineering: Systems degree program in the ECE division.                 \\
 GSI               &  Graduate Student Instructor.                                                        \\
 GSRA              &  Graduate Student Research Assistant.                                                \\
 HKN               &  Eta Kappa Nu.  Student honor society for computer and electrical engineers          \\
 IEEE              &  Institute of Electrical and Electronics Engineers.  Professional society.           \\
 ITCS              &  Information Technology Central Services.  Computing services for all UofM.          \\
 ITS               &  Intelligent Transportation Systems                                                  \\
 RTCL              &  Real Time Computing Lab.  Part of CSE                                               \\
 SSE               &  Systems Science and Engineering.  Also called EECS Systems Lab.  Part of ECE.       \\
 SSL               &  Software Systems Lab.  Part of CSE.                                                 \\
 UofM              &  University of Michigan                                                              \\
\hline
\hline
\end{tabular}
\end{center}
\section{Getting to Know the Right People}
\label{sec-15}

Starting graduate school can feel like being run through a maze.  You have to introduce yourself to at least half a dozen different people, run around to various offices on campus, and sign many pieces of paper.

\begin{itemize}
\item Introduce yourself to Dawn Freysinger, the CSE Division Graduate Student Coordinator.  Her office is in room 3909A CSE.  Dawn can tell you who has been assigned to be your academic advisor.  If you don't already have the following items, you can pick them up here:  requirements for the Master’s and Ph.D. programs, course schedule, instruction sheet for course registration, request for DCO account form.
\item Make an appointment to see your advisor.  Prepare for this by reading the requirements for your chosen degree program as well as the course schedule for the upcoming year.  Try to have some idea of what courses you want to take during your first two semesters (peer counseling can help with this).
\item If you have been offered a position as a GSI or GSRA, or if you want a job as a grader, introduce yourself to Karen Liska in 3709 CSE.  She handles much of the paperwork involving those positions. If you are a GSI, she will give you your staff ID card.  If you are a GSRA, your lab administrator will give you your staff ID card.
\item If you have been given financial support, or want to apply for it, introduce yourself to Dawn Freysinger in room 3909A CSE.
\item If you are interested in AI, introduce yourself to Cindy Watts in room 3816 CSE.
\item If you are interested in Software or Real-Time Computing, introduce yourself to Mindy LaRocca in room 4820 CSE.
\item If you are interested in architecture or hardware, introduce yourself to Denise DuPrie in room 4824 CSE.
\end{itemize}
\subsection{Peer Counseling}
\label{sec-15_1}

CSEG hosts two peer counseling sessions, where a few current graduate
students can answer questions about classes, research, or graduate
life in general.  The first session will be held the week before
classes begin, and the second will be held the first day of class
before CSE orientation.  Attend the first session if you can, since
you must be registered for classes before the second session.

And of course feel free to contact any member of the CSEG board with
any questions that have not been fully answered in this Guide.

\end{document}
